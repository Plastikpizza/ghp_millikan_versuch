\section*{Versuchsbeschreibung}
Der Millikan-Versuch besteht aus zwei Platten die in einem gewissen Abstand voneinander übereinander angeordnet sind. Von oben werden Öltropfen zwischen die Platten gesprüht. Durch ein Mikroskop und eine Skala hinter den Platten ist es Möglich die Fallgeschwindigkeit der Öltröpfchen zu messen. Danach wird an den Platten eine Spannung angelegt so das diese wie ein Kondensator wirken und ein elektrisches Feld entsteht. Wenn die Kraftauswirkung des Feldes auf die Öltröpfchen stark genug ist, steigen die Tröpfchen wieder und man kann die Steiggeschwindigkeit messen. Mit diesen beiden Werten ist es möglich den Radius der Tröpfchen auszurechnen. Durch weitere Berechnungen lässt sich die Elementarladung der Tröpfchen bestimmen.